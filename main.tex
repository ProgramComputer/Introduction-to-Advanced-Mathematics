%day 1 week 1
\documentclass{article}
\usepackage{amsmath}
\usepackage{amssymb}
\usepackage{amsfonts}
\usepackage{amstext}
\usepackage{amsthm}
\usepackage{mathrsfs}

\author{Paul Chiramel}
\date{01/09/23}
\newcommand\vertarrowbox[3][6ex]{%
  \begin{array}[t]{@{}c@{}} #2 \\
  \left\uparrow\vcenter{\hrule height #1}\right.\kern-\nulldelimiterspace\\
  \makebox[0pt]{\scriptsize#3}
  \end{array}%
}
\begin{document}
\section{set}


\underline{def}: a set is a collection of objects . objects that form a set are called elements of the set.

\noindent\underline{Notation}: we use capital letters to denote sets. We use lower case letters to denote elements.




\subsection{Explicitly described sets:}

$\{ 1,2,-3,1/2 \} = S $
$\{1,3,\{1,2\}\}\} = T$
$1 \in S$,
$5 \notin S$
\\\underline{Notation}: $a \in A$ reads " a is a n elements of A", "a belongs to A" or " a lies in A" %TODO
$b \notin A$ reads "b is not an element of A"
\\
$2 \notin T, \{1,2\} \in T$\\
\noindent\underline{def}: The cardinality of a set S is the number of elements in S.

\noindent\underline{Notation}: $\vert S\vert$
For how we will only look at the cardinalities of sets with finitely many elements.
\\
$|S| = 4$
\\
$|T| = 3$

\subsection{Standard reserved names for sets:}

$\mathbb{N}  =$ set of natural numbers $= \{1,2,3,4,.....\}$

\noindent$\mathbb{Z} =$ set of integers $= \{0,1,2,3,.....\}$

\noindent$\mathbb{Q} =$ set of rational numbers

\noindent$\mathbb{R} =$ set of real numbers

\noindent$\mathbb{C} =$ set of complex numbers

\noindent$\emptyset = \{\}$ - set with no elements
- the empty set
$\|\emptyset| = 0$

Ex : $A = \{\emptyset, \{\emptyset\}\}$

\indent\indent$|A| = 2$

\indent\indent$\emptyset \in A$

\indent\indent$\{\emptyset\} \in A$

\indent\indent$|\emptyset| = 1$



\subsection{Sets described by a property:}

$S = \{ x \in U \vert\vertarrowbox{p(x)}{the condition x must satisfy}\}$
Here U in some fixed set\\
\underline{Example}

\noindent$S = \{ x \in \mathbb{N} \vert x^{2} = 9\} = \{3\}$
%\[x^{2} = 9\]
\\
$T = \{ x \in \mathbb{R} \vert x^{2} = 9\} = \{3,-3\}$

\subsection{Sets described by a generating formula:}
$S = \{\vertarrowbox{\mathscr{F}(x)}{ 
     formula
     (an expression in X)
    
    } \vert x \in \mathbb{R}\}$   where $\mathbb{R}$ in some fixed set

\noindent$\mathbb{Q}=\{\frac{a}{b} \vert a \in \mathbb{Z},  b \in \mathbb{N}\}$

\noindent$\mathbb{C} = \{a+bi\vert a,b\in \mathbb{R}\}$ where $i = \sqrt{-1}$
\end{document} 